\documentclass[12pt]{article}
\usepackage[ansinew]{inputenc}
\topmargin=-0.05truecm\relax
 \textheight 23.75cm % 8.6in % 23.5cm
 \textwidth 16.0cm % 5.8in % 15.0cm
\renewcommand{\baselinestretch}{1.2}
\oddsidemargin=0.05truecm\relax%
\evensidemargin=0.05truecm\relax%
 \voffset-1.25cm
 \baselineskip15pt
 \usepackage[dvips]{color}
\begin{document}

\begin{center}
 {\Large\bf }
\vskip 0.2cm

\subsection* {TTT-SiZer: A graphic tool for aging trends recognition}
\subsubsection*{Revision for {\em Reliability Engineering and System Safety}}
\vskip 0.2cm

{\bf February 2020}
\end{center}
\vskip 0.5cm

\date{\ }

\noindent Dear Editor, \vskip 0.3cm

\noindent Please find attached the revision of our paper, ``TTT-SiZer: A graphic tool for aging trends recognition'' by M.L. G\'amiz, R. Nozal-Can\~adas and R. Raya-Miranda.
\bigskip

Thank you for reviewing our paper and the valuable feedback from the Reviewers. We have worked on the paper following your suggestions and comments and produced an improved version of the paper as well as response letters below for the Reviewers.

\bigskip

We hope you will find suitable our revision. 
\bigskip

\vskip 0.3cm
\noindent Yours sincerely,

M.L. G\'amiz

\vskip 0.5cm
%%%%%%%%%%%%%%%%%%%%%%%%%%%%%%%%%%%%   RESPUESTAS
\newpage
\date{\ }
{\centerline {\bf \large Reply letter to the Associate Editor}}
\vskip 1cm
 

\begin{quote}
\emph{ Please revise the paper by addressing the issues a bit more. Thanks.}

\end{quote}


\noindent {\bf Response:} Thank you for reviewing our paper and the valuable feedback from the Reviewers.
%Following your suggestions, we have tried to make the paper more relevant to this journal. To enhance the interest of practitioners in reliability engineering we have included more real examples in this field to emphasize the practical usefulness of the methods. We have worked hard on the introduction to make it clear that both, the topic discussed in this paper (aging) as well as the statistical tool considered (TTT transform) have been recurrent topics in reliability
%analysis (theory and practice) during the last decades.
%
%Aging classes and the TTT curve are still nowadays catching the interest of researchers and practitioners in the area, as can be deduced from the extensive recent bibliography. To highlight this point we have added a considerable number of updated references that are conveniently quoted in the text.
%
%The main purpose of this paper is to bring modern techniques based on nonparametric statistics into the statistical toolbox of reliability analysis, which is not so frequent
%among reliability engineers compared to parametric methods. To do it we introduce a
%new and powerful tool to be used with data before looking for a stochastic (parametric)
%model.
%Following the suggestions of the Associate Editor and Reviewers, we have modified the structure of the manuscript and some mathematical developments have been sketched and moved to an Appendix section added at the end of the manuscript. We have reduced to the
%minimum the mathematical description of the methods.
%We hope this new writing style is considered appropriate for the journal.

%%%%%%%%%%%%%%%%%%%%%%%%%%%%%%%%%%%%%%%%%%%%%%%%%%%%%%%%%%%%
\newpage
%%%%%%%%%%%%%%%%%%%%%%%%%%%%%%%%%%%%%%%%%%%%%%%%%%%%%%%%%%%%%%%%%%%%%%%%%%%%%%%%%%%%%%%%%%%%%%%%%%%%%%%%%
{\centerline {\bf \large Reply letter to Reviewer \#1}}
\vskip 1cm
\noindent Dear Reviewer, \vskip 0.5cm
\noindent Thank you very much for your comments regarding our paper. We have prepared a revised version of the paper following your comments and suggestions, which are included below using italic font for your easy.

\begin{quote}
	\emph{{\bf Review report}\\
The revision is done without much care. The author addressed the most of the suggestions. Still there is minor correction needed. For
example, the newly added reference Zaretala et al. (2018) not discussed in the text. There are several typos still exist.  for example, put a
comma after equation (1). Please correct all these mistakes.
\end{quote}

\noindent Thank you for your recommendation for publishing  subject to major revision. \\
\noindent We respond below in detail to your Major and Minor Comments.

\subsection*{Major Comments}

\begin{quote}
	\emph{(1) In Definition 1.1, The conditional survival function is obtained; conditional on what? one need to specify. Also, Let X a lifetime should
		be Let X be a lifetime random variable )
	}
\end{quote}


\newpage
%%%%%%%%%%%%%%%%%%%%%%%%%%%%%%%%%%%%%%%%%%%%%%%%%%%%%%%%%%%%%%%%%%%%%%%%%%%%%%%%%%%%%%%%%%%%%%%%%%%%%%%%%
{\centerline {\bf \large Reply letter to Reviewer \#4}}
\vskip 1cm
\noindent Dear Reviewer, \vskip 0.5cm

\noindent Thank you very much for your comments regarding our paper. We have prepared a revised version of the paper following your comments and suggestions, which are included below using italic font for your easy.

\begin{quote}
	\emph{ {\bf Review report:} \\
		The paper aims at presenting a new graphic tool to test aging trends based on life data. Although quite a bit of results are shown, I feel that the paper is not suitable for this journal. The focus and analysis are about the statistical properties, and the results are of much more interest for different journals than reliability engineering. The authors probably could rewrite it with more examples and presenting more engineering insights from the study.  Analytical results could be put in an appendix if they are of great importance. The authors should also carry out a more extensive literature search and discuss more recent and related works. 	In particular, works in reliability engineering journals should be discussed to make the paper more relevant. I see that the references are also rather old, and there is only one references from last 3-4 years.}
\end{quote}

{\bf Response:} 
The motivation of this work is to study a classical and crucial issue in reliability analysis such as aging.
For this, we present a powerful and useful graphical tool that can be beneficial from a practical point of view.
To demonstrate the strength of our method it is necessary to study its theoretical statistical properties. We understand that the theoretical support of a new methodological proposal is important for the quality standards of the journal. In addition, the extensive study
of simulation and the applications with real cases that we present demonstrate that our proposal is interesting and has a huge potential for reliability engineering in practice.

Following your suggestion and the Editor's comments, we have changed the structure of the paper in order to make it more readable. Some mathematical explanations, which could be jumped in a first read, have been moved to the Appendix.
The introduction section has been rewritten in the new version of the paper. An extensive literature review is provided including up-to-date works published (some of them in the last year) in several reliability engineering journals. 

 Finally we provide the necessary computer support for the implementation of the method. In this respect, at the moment we are developing a specialized R package that will be available in brief for the interested practitioners.



 We believe that our proposal is interesting and worthwhile from a theoretical and practical point of view and we hope that in its current version it is found to fit well the profile and standards of Reliability Engineering and System Safety. 


\end{document}
